
% Default to the notebook output style

    


% Inherit from the specified cell style.




    
\documentclass[11pt]{article}

    
    
    \usepackage[T1]{fontenc}
    % Nicer default font (+ math font) than Computer Modern for most use cases
    \usepackage{mathpazo}

    % Basic figure setup, for now with no caption control since it's done
    % automatically by Pandoc (which extracts ![](path) syntax from Markdown).
    \usepackage{graphicx}
    % We will generate all images so they have a width \maxwidth. This means
    % that they will get their normal width if they fit onto the page, but
    % are scaled down if they would overflow the margins.
    \makeatletter
    \def\maxwidth{\ifdim\Gin@nat@width>\linewidth\linewidth
    \else\Gin@nat@width\fi}
    \makeatother
    \let\Oldincludegraphics\includegraphics
    % Set max figure width to be 80% of text width, for now hardcoded.
    \renewcommand{\includegraphics}[1]{\Oldincludegraphics[width=.8\maxwidth]{#1}}
    % Ensure that by default, figures have no caption (until we provide a
    % proper Figure object with a Caption API and a way to capture that
    % in the conversion process - todo).
    \usepackage{caption}
    \DeclareCaptionLabelFormat{nolabel}{}
    \captionsetup{labelformat=nolabel}

    \usepackage{adjustbox} % Used to constrain images to a maximum size 
    \usepackage{xcolor} % Allow colors to be defined
    \usepackage{enumerate} % Needed for markdown enumerations to work
    \usepackage{geometry} % Used to adjust the document margins
    \usepackage{amsmath} % Equations
    \usepackage{amssymb} % Equations
    \usepackage{textcomp} % defines textquotesingle
    % Hack from http://tex.stackexchange.com/a/47451/13684:
    \AtBeginDocument{%
        \def\PYZsq{\textquotesingle}% Upright quotes in Pygmentized code
    }
    \usepackage{upquote} % Upright quotes for verbatim code
    \usepackage{eurosym} % defines \euro
    \usepackage[mathletters]{ucs} % Extended unicode (utf-8) support
    \usepackage[utf8x]{inputenc} % Allow utf-8 characters in the tex document
    \usepackage{fancyvrb} % verbatim replacement that allows latex
    \usepackage{grffile} % extends the file name processing of package graphics 
                         % to support a larger range 
    % The hyperref package gives us a pdf with properly built
    % internal navigation ('pdf bookmarks' for the table of contents,
    % internal cross-reference links, web links for URLs, etc.)
    \usepackage{hyperref}
    \usepackage{longtable} % longtable support required by pandoc >1.10
    \usepackage{booktabs}  % table support for pandoc > 1.12.2
    \usepackage[inline]{enumitem} % IRkernel/repr support (it uses the enumerate* environment)
    \usepackage[normalem]{ulem} % ulem is needed to support strikethroughs (\sout)
                                % normalem makes italics be italics, not underlines
    

    
    
    % Colors for the hyperref package
    \definecolor{urlcolor}{rgb}{0,.145,.698}
    \definecolor{linkcolor}{rgb}{.71,0.21,0.01}
    \definecolor{citecolor}{rgb}{.12,.54,.11}

    % ANSI colors
    \definecolor{ansi-black}{HTML}{3E424D}
    \definecolor{ansi-black-intense}{HTML}{282C36}
    \definecolor{ansi-red}{HTML}{E75C58}
    \definecolor{ansi-red-intense}{HTML}{B22B31}
    \definecolor{ansi-green}{HTML}{00A250}
    \definecolor{ansi-green-intense}{HTML}{007427}
    \definecolor{ansi-yellow}{HTML}{DDB62B}
    \definecolor{ansi-yellow-intense}{HTML}{B27D12}
    \definecolor{ansi-blue}{HTML}{208FFB}
    \definecolor{ansi-blue-intense}{HTML}{0065CA}
    \definecolor{ansi-magenta}{HTML}{D160C4}
    \definecolor{ansi-magenta-intense}{HTML}{A03196}
    \definecolor{ansi-cyan}{HTML}{60C6C8}
    \definecolor{ansi-cyan-intense}{HTML}{258F8F}
    \definecolor{ansi-white}{HTML}{C5C1B4}
    \definecolor{ansi-white-intense}{HTML}{A1A6B2}

    % commands and environments needed by pandoc snippets
    % extracted from the output of `pandoc -s`
    \providecommand{\tightlist}{%
      \setlength{\itemsep}{0pt}\setlength{\parskip}{0pt}}
    \DefineVerbatimEnvironment{Highlighting}{Verbatim}{commandchars=\\\{\}}
    % Add ',fontsize=\small' for more characters per line
    \newenvironment{Shaded}{}{}
    \newcommand{\KeywordTok}[1]{\textcolor[rgb]{0.00,0.44,0.13}{\textbf{{#1}}}}
    \newcommand{\DataTypeTok}[1]{\textcolor[rgb]{0.56,0.13,0.00}{{#1}}}
    \newcommand{\DecValTok}[1]{\textcolor[rgb]{0.25,0.63,0.44}{{#1}}}
    \newcommand{\BaseNTok}[1]{\textcolor[rgb]{0.25,0.63,0.44}{{#1}}}
    \newcommand{\FloatTok}[1]{\textcolor[rgb]{0.25,0.63,0.44}{{#1}}}
    \newcommand{\CharTok}[1]{\textcolor[rgb]{0.25,0.44,0.63}{{#1}}}
    \newcommand{\StringTok}[1]{\textcolor[rgb]{0.25,0.44,0.63}{{#1}}}
    \newcommand{\CommentTok}[1]{\textcolor[rgb]{0.38,0.63,0.69}{\textit{{#1}}}}
    \newcommand{\OtherTok}[1]{\textcolor[rgb]{0.00,0.44,0.13}{{#1}}}
    \newcommand{\AlertTok}[1]{\textcolor[rgb]{1.00,0.00,0.00}{\textbf{{#1}}}}
    \newcommand{\FunctionTok}[1]{\textcolor[rgb]{0.02,0.16,0.49}{{#1}}}
    \newcommand{\RegionMarkerTok}[1]{{#1}}
    \newcommand{\ErrorTok}[1]{\textcolor[rgb]{1.00,0.00,0.00}{\textbf{{#1}}}}
    \newcommand{\NormalTok}[1]{{#1}}
    
    % Additional commands for more recent versions of Pandoc
    \newcommand{\ConstantTok}[1]{\textcolor[rgb]{0.53,0.00,0.00}{{#1}}}
    \newcommand{\SpecialCharTok}[1]{\textcolor[rgb]{0.25,0.44,0.63}{{#1}}}
    \newcommand{\VerbatimStringTok}[1]{\textcolor[rgb]{0.25,0.44,0.63}{{#1}}}
    \newcommand{\SpecialStringTok}[1]{\textcolor[rgb]{0.73,0.40,0.53}{{#1}}}
    \newcommand{\ImportTok}[1]{{#1}}
    \newcommand{\DocumentationTok}[1]{\textcolor[rgb]{0.73,0.13,0.13}{\textit{{#1}}}}
    \newcommand{\AnnotationTok}[1]{\textcolor[rgb]{0.38,0.63,0.69}{\textbf{\textit{{#1}}}}}
    \newcommand{\CommentVarTok}[1]{\textcolor[rgb]{0.38,0.63,0.69}{\textbf{\textit{{#1}}}}}
    \newcommand{\VariableTok}[1]{\textcolor[rgb]{0.10,0.09,0.49}{{#1}}}
    \newcommand{\ControlFlowTok}[1]{\textcolor[rgb]{0.00,0.44,0.13}{\textbf{{#1}}}}
    \newcommand{\OperatorTok}[1]{\textcolor[rgb]{0.40,0.40,0.40}{{#1}}}
    \newcommand{\BuiltInTok}[1]{{#1}}
    \newcommand{\ExtensionTok}[1]{{#1}}
    \newcommand{\PreprocessorTok}[1]{\textcolor[rgb]{0.74,0.48,0.00}{{#1}}}
    \newcommand{\AttributeTok}[1]{\textcolor[rgb]{0.49,0.56,0.16}{{#1}}}
    \newcommand{\InformationTok}[1]{\textcolor[rgb]{0.38,0.63,0.69}{\textbf{\textit{{#1}}}}}
    \newcommand{\WarningTok}[1]{\textcolor[rgb]{0.38,0.63,0.69}{\textbf{\textit{{#1}}}}}
    
    
    % Define a nice break command that doesn't care if a line doesn't already
    % exist.
    \def\br{\hspace*{\fill} \\* }
    % Math Jax compatability definitions
    \def\gt{>}
    \def\lt{<}
    % Document parameters
    \title{Chap06}
    
    
    

    % Pygments definitions
    
\makeatletter
\def\PY@reset{\let\PY@it=\relax \let\PY@bf=\relax%
    \let\PY@ul=\relax \let\PY@tc=\relax%
    \let\PY@bc=\relax \let\PY@ff=\relax}
\def\PY@tok#1{\csname PY@tok@#1\endcsname}
\def\PY@toks#1+{\ifx\relax#1\empty\else%
    \PY@tok{#1}\expandafter\PY@toks\fi}
\def\PY@do#1{\PY@bc{\PY@tc{\PY@ul{%
    \PY@it{\PY@bf{\PY@ff{#1}}}}}}}
\def\PY#1#2{\PY@reset\PY@toks#1+\relax+\PY@do{#2}}

\expandafter\def\csname PY@tok@w\endcsname{\def\PY@tc##1{\textcolor[rgb]{0.73,0.73,0.73}{##1}}}
\expandafter\def\csname PY@tok@c\endcsname{\let\PY@it=\textit\def\PY@tc##1{\textcolor[rgb]{0.25,0.50,0.50}{##1}}}
\expandafter\def\csname PY@tok@cp\endcsname{\def\PY@tc##1{\textcolor[rgb]{0.74,0.48,0.00}{##1}}}
\expandafter\def\csname PY@tok@k\endcsname{\let\PY@bf=\textbf\def\PY@tc##1{\textcolor[rgb]{0.00,0.50,0.00}{##1}}}
\expandafter\def\csname PY@tok@kp\endcsname{\def\PY@tc##1{\textcolor[rgb]{0.00,0.50,0.00}{##1}}}
\expandafter\def\csname PY@tok@kt\endcsname{\def\PY@tc##1{\textcolor[rgb]{0.69,0.00,0.25}{##1}}}
\expandafter\def\csname PY@tok@o\endcsname{\def\PY@tc##1{\textcolor[rgb]{0.40,0.40,0.40}{##1}}}
\expandafter\def\csname PY@tok@ow\endcsname{\let\PY@bf=\textbf\def\PY@tc##1{\textcolor[rgb]{0.67,0.13,1.00}{##1}}}
\expandafter\def\csname PY@tok@nb\endcsname{\def\PY@tc##1{\textcolor[rgb]{0.00,0.50,0.00}{##1}}}
\expandafter\def\csname PY@tok@nf\endcsname{\def\PY@tc##1{\textcolor[rgb]{0.00,0.00,1.00}{##1}}}
\expandafter\def\csname PY@tok@nc\endcsname{\let\PY@bf=\textbf\def\PY@tc##1{\textcolor[rgb]{0.00,0.00,1.00}{##1}}}
\expandafter\def\csname PY@tok@nn\endcsname{\let\PY@bf=\textbf\def\PY@tc##1{\textcolor[rgb]{0.00,0.00,1.00}{##1}}}
\expandafter\def\csname PY@tok@ne\endcsname{\let\PY@bf=\textbf\def\PY@tc##1{\textcolor[rgb]{0.82,0.25,0.23}{##1}}}
\expandafter\def\csname PY@tok@nv\endcsname{\def\PY@tc##1{\textcolor[rgb]{0.10,0.09,0.49}{##1}}}
\expandafter\def\csname PY@tok@no\endcsname{\def\PY@tc##1{\textcolor[rgb]{0.53,0.00,0.00}{##1}}}
\expandafter\def\csname PY@tok@nl\endcsname{\def\PY@tc##1{\textcolor[rgb]{0.63,0.63,0.00}{##1}}}
\expandafter\def\csname PY@tok@ni\endcsname{\let\PY@bf=\textbf\def\PY@tc##1{\textcolor[rgb]{0.60,0.60,0.60}{##1}}}
\expandafter\def\csname PY@tok@na\endcsname{\def\PY@tc##1{\textcolor[rgb]{0.49,0.56,0.16}{##1}}}
\expandafter\def\csname PY@tok@nt\endcsname{\let\PY@bf=\textbf\def\PY@tc##1{\textcolor[rgb]{0.00,0.50,0.00}{##1}}}
\expandafter\def\csname PY@tok@nd\endcsname{\def\PY@tc##1{\textcolor[rgb]{0.67,0.13,1.00}{##1}}}
\expandafter\def\csname PY@tok@s\endcsname{\def\PY@tc##1{\textcolor[rgb]{0.73,0.13,0.13}{##1}}}
\expandafter\def\csname PY@tok@sd\endcsname{\let\PY@it=\textit\def\PY@tc##1{\textcolor[rgb]{0.73,0.13,0.13}{##1}}}
\expandafter\def\csname PY@tok@si\endcsname{\let\PY@bf=\textbf\def\PY@tc##1{\textcolor[rgb]{0.73,0.40,0.53}{##1}}}
\expandafter\def\csname PY@tok@se\endcsname{\let\PY@bf=\textbf\def\PY@tc##1{\textcolor[rgb]{0.73,0.40,0.13}{##1}}}
\expandafter\def\csname PY@tok@sr\endcsname{\def\PY@tc##1{\textcolor[rgb]{0.73,0.40,0.53}{##1}}}
\expandafter\def\csname PY@tok@ss\endcsname{\def\PY@tc##1{\textcolor[rgb]{0.10,0.09,0.49}{##1}}}
\expandafter\def\csname PY@tok@sx\endcsname{\def\PY@tc##1{\textcolor[rgb]{0.00,0.50,0.00}{##1}}}
\expandafter\def\csname PY@tok@m\endcsname{\def\PY@tc##1{\textcolor[rgb]{0.40,0.40,0.40}{##1}}}
\expandafter\def\csname PY@tok@gh\endcsname{\let\PY@bf=\textbf\def\PY@tc##1{\textcolor[rgb]{0.00,0.00,0.50}{##1}}}
\expandafter\def\csname PY@tok@gu\endcsname{\let\PY@bf=\textbf\def\PY@tc##1{\textcolor[rgb]{0.50,0.00,0.50}{##1}}}
\expandafter\def\csname PY@tok@gd\endcsname{\def\PY@tc##1{\textcolor[rgb]{0.63,0.00,0.00}{##1}}}
\expandafter\def\csname PY@tok@gi\endcsname{\def\PY@tc##1{\textcolor[rgb]{0.00,0.63,0.00}{##1}}}
\expandafter\def\csname PY@tok@gr\endcsname{\def\PY@tc##1{\textcolor[rgb]{1.00,0.00,0.00}{##1}}}
\expandafter\def\csname PY@tok@ge\endcsname{\let\PY@it=\textit}
\expandafter\def\csname PY@tok@gs\endcsname{\let\PY@bf=\textbf}
\expandafter\def\csname PY@tok@gp\endcsname{\let\PY@bf=\textbf\def\PY@tc##1{\textcolor[rgb]{0.00,0.00,0.50}{##1}}}
\expandafter\def\csname PY@tok@go\endcsname{\def\PY@tc##1{\textcolor[rgb]{0.53,0.53,0.53}{##1}}}
\expandafter\def\csname PY@tok@gt\endcsname{\def\PY@tc##1{\textcolor[rgb]{0.00,0.27,0.87}{##1}}}
\expandafter\def\csname PY@tok@err\endcsname{\def\PY@bc##1{\setlength{\fboxsep}{0pt}\fcolorbox[rgb]{1.00,0.00,0.00}{1,1,1}{\strut ##1}}}
\expandafter\def\csname PY@tok@kc\endcsname{\let\PY@bf=\textbf\def\PY@tc##1{\textcolor[rgb]{0.00,0.50,0.00}{##1}}}
\expandafter\def\csname PY@tok@kd\endcsname{\let\PY@bf=\textbf\def\PY@tc##1{\textcolor[rgb]{0.00,0.50,0.00}{##1}}}
\expandafter\def\csname PY@tok@kn\endcsname{\let\PY@bf=\textbf\def\PY@tc##1{\textcolor[rgb]{0.00,0.50,0.00}{##1}}}
\expandafter\def\csname PY@tok@kr\endcsname{\let\PY@bf=\textbf\def\PY@tc##1{\textcolor[rgb]{0.00,0.50,0.00}{##1}}}
\expandafter\def\csname PY@tok@bp\endcsname{\def\PY@tc##1{\textcolor[rgb]{0.00,0.50,0.00}{##1}}}
\expandafter\def\csname PY@tok@fm\endcsname{\def\PY@tc##1{\textcolor[rgb]{0.00,0.00,1.00}{##1}}}
\expandafter\def\csname PY@tok@vc\endcsname{\def\PY@tc##1{\textcolor[rgb]{0.10,0.09,0.49}{##1}}}
\expandafter\def\csname PY@tok@vg\endcsname{\def\PY@tc##1{\textcolor[rgb]{0.10,0.09,0.49}{##1}}}
\expandafter\def\csname PY@tok@vi\endcsname{\def\PY@tc##1{\textcolor[rgb]{0.10,0.09,0.49}{##1}}}
\expandafter\def\csname PY@tok@vm\endcsname{\def\PY@tc##1{\textcolor[rgb]{0.10,0.09,0.49}{##1}}}
\expandafter\def\csname PY@tok@sa\endcsname{\def\PY@tc##1{\textcolor[rgb]{0.73,0.13,0.13}{##1}}}
\expandafter\def\csname PY@tok@sb\endcsname{\def\PY@tc##1{\textcolor[rgb]{0.73,0.13,0.13}{##1}}}
\expandafter\def\csname PY@tok@sc\endcsname{\def\PY@tc##1{\textcolor[rgb]{0.73,0.13,0.13}{##1}}}
\expandafter\def\csname PY@tok@dl\endcsname{\def\PY@tc##1{\textcolor[rgb]{0.73,0.13,0.13}{##1}}}
\expandafter\def\csname PY@tok@s2\endcsname{\def\PY@tc##1{\textcolor[rgb]{0.73,0.13,0.13}{##1}}}
\expandafter\def\csname PY@tok@sh\endcsname{\def\PY@tc##1{\textcolor[rgb]{0.73,0.13,0.13}{##1}}}
\expandafter\def\csname PY@tok@s1\endcsname{\def\PY@tc##1{\textcolor[rgb]{0.73,0.13,0.13}{##1}}}
\expandafter\def\csname PY@tok@mb\endcsname{\def\PY@tc##1{\textcolor[rgb]{0.40,0.40,0.40}{##1}}}
\expandafter\def\csname PY@tok@mf\endcsname{\def\PY@tc##1{\textcolor[rgb]{0.40,0.40,0.40}{##1}}}
\expandafter\def\csname PY@tok@mh\endcsname{\def\PY@tc##1{\textcolor[rgb]{0.40,0.40,0.40}{##1}}}
\expandafter\def\csname PY@tok@mi\endcsname{\def\PY@tc##1{\textcolor[rgb]{0.40,0.40,0.40}{##1}}}
\expandafter\def\csname PY@tok@il\endcsname{\def\PY@tc##1{\textcolor[rgb]{0.40,0.40,0.40}{##1}}}
\expandafter\def\csname PY@tok@mo\endcsname{\def\PY@tc##1{\textcolor[rgb]{0.40,0.40,0.40}{##1}}}
\expandafter\def\csname PY@tok@ch\endcsname{\let\PY@it=\textit\def\PY@tc##1{\textcolor[rgb]{0.25,0.50,0.50}{##1}}}
\expandafter\def\csname PY@tok@cm\endcsname{\let\PY@it=\textit\def\PY@tc##1{\textcolor[rgb]{0.25,0.50,0.50}{##1}}}
\expandafter\def\csname PY@tok@cpf\endcsname{\let\PY@it=\textit\def\PY@tc##1{\textcolor[rgb]{0.25,0.50,0.50}{##1}}}
\expandafter\def\csname PY@tok@c1\endcsname{\let\PY@it=\textit\def\PY@tc##1{\textcolor[rgb]{0.25,0.50,0.50}{##1}}}
\expandafter\def\csname PY@tok@cs\endcsname{\let\PY@it=\textit\def\PY@tc##1{\textcolor[rgb]{0.25,0.50,0.50}{##1}}}

\def\PYZbs{\char`\\}
\def\PYZus{\char`\_}
\def\PYZob{\char`\{}
\def\PYZcb{\char`\}}
\def\PYZca{\char`\^}
\def\PYZam{\char`\&}
\def\PYZlt{\char`\<}
\def\PYZgt{\char`\>}
\def\PYZsh{\char`\#}
\def\PYZpc{\char`\%}
\def\PYZdl{\char`\$}
\def\PYZhy{\char`\-}
\def\PYZsq{\char`\'}
\def\PYZdq{\char`\"}
\def\PYZti{\char`\~}
% for compatibility with earlier versions
\def\PYZat{@}
\def\PYZlb{[}
\def\PYZrb{]}
\makeatother


    % Exact colors from NB
    \definecolor{incolor}{rgb}{0.0, 0.0, 0.5}
    \definecolor{outcolor}{rgb}{0.545, 0.0, 0.0}



    
    % Prevent overflowing lines due to hard-to-break entities
    \sloppy 
    % Setup hyperref package
    \hypersetup{
      breaklinks=true,  % so long urls are correctly broken across lines
      colorlinks=true,
      urlcolor=urlcolor,
      linkcolor=linkcolor,
      citecolor=citecolor,
      }
    % Slightly bigger margins than the latex defaults
    
    \geometry{verbose,tmargin=1in,bmargin=1in,lmargin=1in,rmargin=1in}
    
    

    \begin{document}
    
    
    \maketitle
    
    

    
    \section{Group Segmentation: Lending
Club}\label{group-segmentation-lending-club}

Clustering allows us to identify groups whose members are similar to
each other and dissimilar to members of other groups. This gives us a
powerful instrument that makes it possible to find different consumer
groups for online trades, user segments for marketing, etc. In this
chapter, we will apply an unsupervised learning solution using
clustering algorithms from the previous chapter to perform group
segmentation.

\subsection{Lending Club Data}\label{lending-club-data}

We will use loan data available from the Lending Club, a US peer-to-peer
lending company, The platform allows borrowers to get loans between
\$1000 and \$40000 for a term of either three or five years.

Investors can browse the loan application and choose to finance the
loans based on the credit history of the borrower, the amount of the
loan and the purpose of the loan.

The data is from years 2007 - 2011 and is publicly available on the
Lending Club website
(https://www.lendingclub.com/info/download-data.action).

We import the required libraries and data first:

    \begin{Verbatim}[commandchars=\\\{\}]
{\color{incolor}In [{\color{incolor}1}]:} \PY{c+c1}{\PYZsh{} Import libraries}
        \PY{l+s+sd}{\PYZsq{}\PYZsq{}\PYZsq{}Main\PYZsq{}\PYZsq{}\PYZsq{}}
        \PY{k+kn}{import} \PY{n+nn}{numpy} \PY{k}{as} \PY{n+nn}{np}
        \PY{k+kn}{import} \PY{n+nn}{pandas} \PY{k}{as} \PY{n+nn}{pd}
        \PY{k+kn}{import} \PY{n+nn}{os}\PY{o}{,} \PY{n+nn}{time}\PY{o}{,} \PY{n+nn}{re}
        \PY{k+kn}{import} \PY{n+nn}{pickle}\PY{o}{,} \PY{n+nn}{gzip}
        
        \PY{l+s+sd}{\PYZsq{}\PYZsq{}\PYZsq{}Data Viz\PYZsq{}\PYZsq{}\PYZsq{}}
        \PY{k+kn}{import} \PY{n+nn}{matplotlib}\PY{n+nn}{.}\PY{n+nn}{pyplot} \PY{k}{as} \PY{n+nn}{plt}
        \PY{k+kn}{import} \PY{n+nn}{seaborn} \PY{k}{as} \PY{n+nn}{sns}
        \PY{n}{color} \PY{o}{=} \PY{n}{sns}\PY{o}{.}\PY{n}{color\PYZus{}palette}\PY{p}{(}\PY{p}{)}
        \PY{k+kn}{import} \PY{n+nn}{matplotlib} \PY{k}{as} \PY{n+nn}{mpl}
        
        \PY{o}{\PYZpc{}}\PY{k}{matplotlib} inline
        
        \PY{l+s+sd}{\PYZsq{}\PYZsq{}\PYZsq{}Data Prep and Model Evaluation\PYZsq{}\PYZsq{}\PYZsq{}}
        \PY{k+kn}{from} \PY{n+nn}{sklearn} \PY{k}{import} \PY{n}{preprocessing} \PY{k}{as} \PY{n}{pp}
        \PY{k+kn}{from} \PY{n+nn}{sklearn}\PY{n+nn}{.}\PY{n+nn}{model\PYZus{}selection} \PY{k}{import} \PY{n}{train\PYZus{}test\PYZus{}split} 
        \PY{k+kn}{from} \PY{n+nn}{sklearn}\PY{n+nn}{.}\PY{n+nn}{metrics} \PY{k}{import} \PY{n}{precision\PYZus{}recall\PYZus{}curve}\PY{p}{,} \PY{n}{average\PYZus{}precision\PYZus{}score}
        \PY{k+kn}{from} \PY{n+nn}{sklearn}\PY{n+nn}{.}\PY{n+nn}{metrics} \PY{k}{import} \PY{n}{roc\PYZus{}curve}\PY{p}{,} \PY{n}{auc}\PY{p}{,} \PY{n}{roc\PYZus{}auc\PYZus{}score}
        
        \PY{l+s+sd}{\PYZsq{}\PYZsq{}\PYZsq{}Algorithms\PYZsq{}\PYZsq{}\PYZsq{}}
        \PY{k+kn}{from} \PY{n+nn}{sklearn}\PY{n+nn}{.}\PY{n+nn}{decomposition} \PY{k}{import} \PY{n}{PCA}
        \PY{k+kn}{from} \PY{n+nn}{sklearn}\PY{n+nn}{.}\PY{n+nn}{cluster} \PY{k}{import} \PY{n}{KMeans}
        \PY{k+kn}{import} \PY{n+nn}{fastcluster}
        \PY{k+kn}{from} \PY{n+nn}{scipy}\PY{n+nn}{.}\PY{n+nn}{cluster}\PY{n+nn}{.}\PY{n+nn}{hierarchy} \PY{k}{import} \PY{n}{dendrogram}\PY{p}{,} \PY{n}{cophenet}\PY{p}{,} \PY{n}{fcluster}
        \PY{k+kn}{from} \PY{n+nn}{scipy}\PY{n+nn}{.}\PY{n+nn}{spatial}\PY{n+nn}{.}\PY{n+nn}{distance} \PY{k}{import} \PY{n}{pdist}
\end{Verbatim}


    \subsubsection{Explore the Data}\label{explore-the-data}

The original data file has 144 columns but most of them are empty and
have little use. We will keep only those features that make sense for
our clustering application. These include among others:

\begin{itemize}
\tightlist
\item
  amount requested
\item
  amount funded
\item
  loan term
\item
  interest rate
\item
  loan grade
\end{itemize}

    \begin{Verbatim}[commandchars=\\\{\}]
{\color{incolor}In [{\color{incolor}2}]:} \PY{c+c1}{\PYZsh{} Load the data}
        \PY{n}{current\PYZus{}path} \PY{o}{=} \PY{n}{os}\PY{o}{.}\PY{n}{getcwd}\PY{p}{(}\PY{p}{)}
        \PY{n}{file} \PY{o}{=} \PY{l+s+s1}{\PYZsq{}}\PY{l+s+se}{\PYZbs{}\PYZbs{}}\PY{l+s+s1}{datasets}\PY{l+s+se}{\PYZbs{}\PYZbs{}}\PY{l+s+s1}{lending\PYZus{}club\PYZus{}data}\PY{l+s+se}{\PYZbs{}\PYZbs{}}\PY{l+s+s1}{LoanStats3a.csv}\PY{l+s+s1}{\PYZsq{}}
        \PY{n}{data} \PY{o}{=} \PY{n}{pd}\PY{o}{.}\PY{n}{read\PYZus{}csv}\PY{p}{(}\PY{n}{current\PYZus{}path} \PY{o}{+} \PY{n}{file}\PY{p}{)}
\end{Verbatim}


    \begin{Verbatim}[commandchars=\\\{\}]
C:\textbackslash{}ProgramData\textbackslash{}Anaconda3\textbackslash{}lib\textbackslash{}site-packages\textbackslash{}IPython\textbackslash{}core\textbackslash{}interactiveshell.py:2785: DtypeWarning: Columns (0,47) have mixed types. Specify dtype option on import or set low\_memory=False.
  interactivity=interactivity, compiler=compiler, result=result)

    \end{Verbatim}

    \begin{Verbatim}[commandchars=\\\{\}]
{\color{incolor}In [{\color{incolor}3}]:} \PY{c+c1}{\PYZsh{} Select columns to keep}
        \PY{n}{columnsToKeep} \PY{o}{=} \PY{p}{[}\PY{l+s+s1}{\PYZsq{}}\PY{l+s+s1}{loan\PYZus{}amnt}\PY{l+s+s1}{\PYZsq{}}\PY{p}{,}\PY{l+s+s1}{\PYZsq{}}\PY{l+s+s1}{funded\PYZus{}amnt}\PY{l+s+s1}{\PYZsq{}}\PY{p}{,}\PY{l+s+s1}{\PYZsq{}}\PY{l+s+s1}{funded\PYZus{}amnt\PYZus{}inv}\PY{l+s+s1}{\PYZsq{}}\PY{p}{,}\PY{l+s+s1}{\PYZsq{}}\PY{l+s+s1}{term}\PY{l+s+s1}{\PYZsq{}}\PY{p}{,} \PYZbs{}
                         \PY{l+s+s1}{\PYZsq{}}\PY{l+s+s1}{int\PYZus{}rate}\PY{l+s+s1}{\PYZsq{}}\PY{p}{,}\PY{l+s+s1}{\PYZsq{}}\PY{l+s+s1}{installment}\PY{l+s+s1}{\PYZsq{}}\PY{p}{,}\PY{l+s+s1}{\PYZsq{}}\PY{l+s+s1}{grade}\PY{l+s+s1}{\PYZsq{}}\PY{p}{,}\PY{l+s+s1}{\PYZsq{}}\PY{l+s+s1}{sub\PYZus{}grade}\PY{l+s+s1}{\PYZsq{}}\PY{p}{,} \PYZbs{}
                         \PY{l+s+s1}{\PYZsq{}}\PY{l+s+s1}{emp\PYZus{}length}\PY{l+s+s1}{\PYZsq{}}\PY{p}{,}\PY{l+s+s1}{\PYZsq{}}\PY{l+s+s1}{home\PYZus{}ownership}\PY{l+s+s1}{\PYZsq{}}\PY{p}{,}\PY{l+s+s1}{\PYZsq{}}\PY{l+s+s1}{annual\PYZus{}inc}\PY{l+s+s1}{\PYZsq{}}\PY{p}{,} \PYZbs{}
                         \PY{l+s+s1}{\PYZsq{}}\PY{l+s+s1}{verification\PYZus{}status}\PY{l+s+s1}{\PYZsq{}}\PY{p}{,}\PY{l+s+s1}{\PYZsq{}}\PY{l+s+s1}{pymnt\PYZus{}plan}\PY{l+s+s1}{\PYZsq{}}\PY{p}{,}\PY{l+s+s1}{\PYZsq{}}\PY{l+s+s1}{purpose}\PY{l+s+s1}{\PYZsq{}}\PY{p}{,} \PYZbs{}
                         \PY{l+s+s1}{\PYZsq{}}\PY{l+s+s1}{addr\PYZus{}state}\PY{l+s+s1}{\PYZsq{}}\PY{p}{,}\PY{l+s+s1}{\PYZsq{}}\PY{l+s+s1}{dti}\PY{l+s+s1}{\PYZsq{}}\PY{p}{,}\PY{l+s+s1}{\PYZsq{}}\PY{l+s+s1}{delinq\PYZus{}2yrs}\PY{l+s+s1}{\PYZsq{}}\PY{p}{,}\PY{l+s+s1}{\PYZsq{}}\PY{l+s+s1}{earliest\PYZus{}cr\PYZus{}line}\PY{l+s+s1}{\PYZsq{}}\PY{p}{,} \PYZbs{}
                         \PY{l+s+s1}{\PYZsq{}}\PY{l+s+s1}{mths\PYZus{}since\PYZus{}last\PYZus{}delinq}\PY{l+s+s1}{\PYZsq{}}\PY{p}{,}\PY{l+s+s1}{\PYZsq{}}\PY{l+s+s1}{mths\PYZus{}since\PYZus{}last\PYZus{}record}\PY{l+s+s1}{\PYZsq{}}\PY{p}{,} \PYZbs{}
                         \PY{l+s+s1}{\PYZsq{}}\PY{l+s+s1}{open\PYZus{}acc}\PY{l+s+s1}{\PYZsq{}}\PY{p}{,}\PY{l+s+s1}{\PYZsq{}}\PY{l+s+s1}{pub\PYZus{}rec}\PY{l+s+s1}{\PYZsq{}}\PY{p}{,}\PY{l+s+s1}{\PYZsq{}}\PY{l+s+s1}{revol\PYZus{}bal}\PY{l+s+s1}{\PYZsq{}}\PY{p}{,}\PY{l+s+s1}{\PYZsq{}}\PY{l+s+s1}{revol\PYZus{}util}\PY{l+s+s1}{\PYZsq{}}\PY{p}{,} \PYZbs{}
                         \PY{l+s+s1}{\PYZsq{}}\PY{l+s+s1}{total\PYZus{}acc}\PY{l+s+s1}{\PYZsq{}}\PY{p}{,}\PY{l+s+s1}{\PYZsq{}}\PY{l+s+s1}{initial\PYZus{}list\PYZus{}status}\PY{l+s+s1}{\PYZsq{}}\PY{p}{,}\PY{l+s+s1}{\PYZsq{}}\PY{l+s+s1}{out\PYZus{}prncp}\PY{l+s+s1}{\PYZsq{}}\PY{p}{,} \PYZbs{}
                         \PY{l+s+s1}{\PYZsq{}}\PY{l+s+s1}{out\PYZus{}prncp\PYZus{}inv}\PY{l+s+s1}{\PYZsq{}}\PY{p}{,}\PY{l+s+s1}{\PYZsq{}}\PY{l+s+s1}{total\PYZus{}pymnt}\PY{l+s+s1}{\PYZsq{}}\PY{p}{,}\PY{l+s+s1}{\PYZsq{}}\PY{l+s+s1}{total\PYZus{}pymnt\PYZus{}inv}\PY{l+s+s1}{\PYZsq{}}\PY{p}{,} \PYZbs{}
                         \PY{l+s+s1}{\PYZsq{}}\PY{l+s+s1}{total\PYZus{}rec\PYZus{}prncp}\PY{l+s+s1}{\PYZsq{}}\PY{p}{,}\PY{l+s+s1}{\PYZsq{}}\PY{l+s+s1}{total\PYZus{}rec\PYZus{}int}\PY{l+s+s1}{\PYZsq{}}\PY{p}{,}\PY{l+s+s1}{\PYZsq{}}\PY{l+s+s1}{total\PYZus{}rec\PYZus{}late\PYZus{}fee}\PY{l+s+s1}{\PYZsq{}}\PY{p}{,} \PYZbs{}
                         \PY{l+s+s1}{\PYZsq{}}\PY{l+s+s1}{recoveries}\PY{l+s+s1}{\PYZsq{}}\PY{p}{,}\PY{l+s+s1}{\PYZsq{}}\PY{l+s+s1}{collection\PYZus{}recovery\PYZus{}fee}\PY{l+s+s1}{\PYZsq{}}\PY{p}{,}\PY{l+s+s1}{\PYZsq{}}\PY{l+s+s1}{last\PYZus{}pymnt\PYZus{}d}\PY{l+s+s1}{\PYZsq{}}\PY{p}{,} \PYZbs{}
                         \PY{l+s+s1}{\PYZsq{}}\PY{l+s+s1}{last\PYZus{}pymnt\PYZus{}amnt}\PY{l+s+s1}{\PYZsq{}}\PY{p}{]}
        
        \PY{n}{data} \PY{o}{=} \PY{n}{data}\PY{o}{.}\PY{n}{loc}\PY{p}{[}\PY{p}{:}\PY{p}{,}\PY{n}{columnsToKeep}\PY{p}{]}
\end{Verbatim}


    \begin{Verbatim}[commandchars=\\\{\}]
{\color{incolor}In [{\color{incolor}5}]:} \PY{n}{data}\PY{o}{.}\PY{n}{head}\PY{p}{(}\PY{p}{)}
\end{Verbatim}


\begin{Verbatim}[commandchars=\\\{\}]
{\color{outcolor}Out[{\color{outcolor}5}]:}    loan\_amnt  funded\_amnt  funded\_amnt\_inv        term int\_rate  installment  \textbackslash{}
        0     5000.0       5000.0           4975.0   36 months   10.65\%       162.87   
        1     2500.0       2500.0           2500.0   60 months   15.27\%        59.83   
        2     2400.0       2400.0           2400.0   36 months   15.96\%        84.33   
        3    10000.0      10000.0          10000.0   36 months   13.49\%       339.31   
        4     3000.0       3000.0           3000.0   60 months   12.69\%        67.79   
        
          grade sub\_grade emp\_length home\_ownership       {\ldots}         out\_prncp\_inv  \textbackslash{}
        0     B        B2  10+ years           RENT       {\ldots}                   0.0   
        1     C        C4   < 1 year           RENT       {\ldots}                   0.0   
        2     C        C5  10+ years           RENT       {\ldots}                   0.0   
        3     C        C1  10+ years           RENT       {\ldots}                   0.0   
        4     B        B5     1 year           RENT       {\ldots}                   0.0   
        
            total\_pymnt total\_pymnt\_inv total\_rec\_prncp total\_rec\_int  \textbackslash{}
        0   5863.155187         5833.84         5000.00        863.16   
        1   1014.530000         1014.53          456.46        435.17   
        2   3005.666844         3005.67         2400.00        605.67   
        3  12231.890000        12231.89        10000.00       2214.92   
        4   4066.908161         4066.91         3000.00       1066.91   
        
           total\_rec\_late\_fee  recoveries collection\_recovery\_fee  last\_pymnt\_d  \textbackslash{}
        0                0.00         0.0                    0.00        Jan-15   
        1                0.00       122.9                    1.11        Apr-13   
        2                0.00         0.0                    0.00        Jun-14   
        3               16.97         0.0                    0.00        Jan-15   
        4                0.00         0.0                    0.00        Jan-17   
        
           last\_pymnt\_amnt  
        0           171.62  
        1           119.66  
        2           649.91  
        3           357.48  
        4            67.30  
        
        [5 rows x 37 columns]
\end{Verbatim}
            
    \subsubsection{Transform String Format to Numerical
Format}\label{transform-string-format-to-numerical-format}

Some of the features (the term of the loan, the interest rate,
employment length, revolving utilization of the borrower - need to be
converted from a string format to a numerical form.

    \begin{Verbatim}[commandchars=\\\{\}]
{\color{incolor}In [{\color{incolor}6}]:} \PY{c+c1}{\PYZsh{} Transform features from string to numeric}
        \PY{k}{for} \PY{n}{i} \PY{o+ow}{in} \PY{p}{[}\PY{l+s+s2}{\PYZdq{}}\PY{l+s+s2}{term}\PY{l+s+s2}{\PYZdq{}}\PY{p}{,}\PY{l+s+s2}{\PYZdq{}}\PY{l+s+s2}{int\PYZus{}rate}\PY{l+s+s2}{\PYZdq{}}\PY{p}{,}\PY{l+s+s2}{\PYZdq{}}\PY{l+s+s2}{emp\PYZus{}length}\PY{l+s+s2}{\PYZdq{}}\PY{p}{,}\PY{l+s+s2}{\PYZdq{}}\PY{l+s+s2}{revol\PYZus{}util}\PY{l+s+s2}{\PYZdq{}}\PY{p}{]}\PY{p}{:}
            \PY{n}{data}\PY{o}{.}\PY{n}{loc}\PY{p}{[}\PY{p}{:}\PY{p}{,}\PY{n}{i}\PY{p}{]} \PY{o}{=} \PYZbs{}
                \PY{n}{data}\PY{o}{.}\PY{n}{loc}\PY{p}{[}\PY{p}{:}\PY{p}{,}\PY{n}{i}\PY{p}{]}\PY{o}{.}\PY{n}{apply}\PY{p}{(}\PY{k}{lambda} \PY{n}{x}\PY{p}{:} \PY{n}{re}\PY{o}{.}\PY{n}{sub}\PY{p}{(}\PY{l+s+s2}{\PYZdq{}}\PY{l+s+s2}{[\PYZca{}0\PYZhy{}9]}\PY{l+s+s2}{\PYZdq{}}\PY{p}{,} \PY{l+s+s2}{\PYZdq{}}\PY{l+s+s2}{\PYZdq{}}\PY{p}{,} \PY{n+nb}{str}\PY{p}{(}\PY{n}{x}\PY{p}{)}\PY{p}{)}\PY{p}{)}
            \PY{n}{data}\PY{o}{.}\PY{n}{loc}\PY{p}{[}\PY{p}{:}\PY{p}{,}\PY{n}{i}\PY{p}{]} \PY{o}{=} \PY{n}{pd}\PY{o}{.}\PY{n}{to\PYZus{}numeric}\PY{p}{(}\PY{n}{data}\PY{o}{.}\PY{n}{loc}\PY{p}{[}\PY{p}{:}\PY{p}{,}\PY{n}{i}\PY{p}{]}\PY{p}{)}
\end{Verbatim}


    \subsubsection{Impute Missing Data}\label{impute-missing-data}

Some features may have missing values. We have to explore the numerical
columns and count NaNs. Then we will impute those NaNs either using a
mean or zero, depending on the feature.

    \begin{Verbatim}[commandchars=\\\{\}]
{\color{incolor}In [{\color{incolor}7}]:} \PY{c+c1}{\PYZsh{} Determine which features are numerical}
        \PY{n}{numericalFeats} \PY{o}{=} \PY{p}{[}\PY{n}{x} \PY{k}{for} \PY{n}{x} \PY{o+ow}{in} \PY{n}{data}\PY{o}{.}\PY{n}{columns} \PY{k}{if} \PY{n}{data}\PY{p}{[}\PY{n}{x}\PY{p}{]}\PY{o}{.}\PY{n}{dtype} \PY{o}{!=} \PY{l+s+s1}{\PYZsq{}}\PY{l+s+s1}{object}\PY{l+s+s1}{\PYZsq{}}\PY{p}{]}
\end{Verbatim}


    \begin{Verbatim}[commandchars=\\\{\}]
{\color{incolor}In [{\color{incolor}8}]:} \PY{c+c1}{\PYZsh{} Display NaNs by feature}
        \PY{n}{nanCounter} \PY{o}{=} \PY{n}{np}\PY{o}{.}\PY{n}{isnan}\PY{p}{(}\PY{n}{data}\PY{o}{.}\PY{n}{loc}\PY{p}{[}\PY{p}{:}\PY{p}{,}\PY{n}{numericalFeats}\PY{p}{]}\PY{p}{)}\PY{o}{.}\PY{n}{sum}\PY{p}{(}\PY{p}{)}
        \PY{n}{nanCounter}
\end{Verbatim}


\begin{Verbatim}[commandchars=\\\{\}]
{\color{outcolor}Out[{\color{outcolor}8}]:} loan\_amnt                      7
        funded\_amnt                    7
        funded\_amnt\_inv                7
        term                           7
        int\_rate                       7
        installment                    7
        emp\_length                  1119
        annual\_inc                    11
        dti                            7
        delinq\_2yrs                   36
        mths\_since\_last\_delinq     26933
        mths\_since\_last\_record     38891
        open\_acc                      36
        pub\_rec                       36
        revol\_bal                      7
        revol\_util                    97
        total\_acc                     36
        out\_prncp                      7
        out\_prncp\_inv                  7
        total\_pymnt                    7
        total\_pymnt\_inv                7
        total\_rec\_prncp                7
        total\_rec\_int                  7
        total\_rec\_late\_fee             7
        recoveries                     7
        collection\_recovery\_fee        7
        last\_pymnt\_amnt                7
        dtype: int64
\end{Verbatim}
            
    Most features have only a few NaNs, but some have a significant portion
of NaN values. We will fill some fo the features with their respective
mean values, and some feature with zeros:

    \begin{Verbatim}[commandchars=\\\{\}]
{\color{incolor}In [{\color{incolor}9}]:} \PY{c+c1}{\PYZsh{} Impute NaNs with mean }
        \PY{n}{fillWithMean} \PY{o}{=} \PY{p}{[}\PY{l+s+s1}{\PYZsq{}}\PY{l+s+s1}{loan\PYZus{}amnt}\PY{l+s+s1}{\PYZsq{}}\PY{p}{,}\PY{l+s+s1}{\PYZsq{}}\PY{l+s+s1}{funded\PYZus{}amnt}\PY{l+s+s1}{\PYZsq{}}\PY{p}{,}\PY{l+s+s1}{\PYZsq{}}\PY{l+s+s1}{funded\PYZus{}amnt\PYZus{}inv}\PY{l+s+s1}{\PYZsq{}}\PY{p}{,}\PY{l+s+s1}{\PYZsq{}}\PY{l+s+s1}{term}\PY{l+s+s1}{\PYZsq{}}\PY{p}{,} \PYZbs{}
                        \PY{l+s+s1}{\PYZsq{}}\PY{l+s+s1}{int\PYZus{}rate}\PY{l+s+s1}{\PYZsq{}}\PY{p}{,}\PY{l+s+s1}{\PYZsq{}}\PY{l+s+s1}{installment}\PY{l+s+s1}{\PYZsq{}}\PY{p}{,}\PY{l+s+s1}{\PYZsq{}}\PY{l+s+s1}{emp\PYZus{}length}\PY{l+s+s1}{\PYZsq{}}\PY{p}{,}\PY{l+s+s1}{\PYZsq{}}\PY{l+s+s1}{annual\PYZus{}inc}\PY{l+s+s1}{\PYZsq{}}\PY{p}{,}\PYZbs{}
                        \PY{l+s+s1}{\PYZsq{}}\PY{l+s+s1}{dti}\PY{l+s+s1}{\PYZsq{}}\PY{p}{,}\PY{l+s+s1}{\PYZsq{}}\PY{l+s+s1}{open\PYZus{}acc}\PY{l+s+s1}{\PYZsq{}}\PY{p}{,}\PY{l+s+s1}{\PYZsq{}}\PY{l+s+s1}{revol\PYZus{}bal}\PY{l+s+s1}{\PYZsq{}}\PY{p}{,}\PY{l+s+s1}{\PYZsq{}}\PY{l+s+s1}{revol\PYZus{}util}\PY{l+s+s1}{\PYZsq{}}\PY{p}{,}\PY{l+s+s1}{\PYZsq{}}\PY{l+s+s1}{total\PYZus{}acc}\PY{l+s+s1}{\PYZsq{}}\PY{p}{,}\PYZbs{}
                        \PY{l+s+s1}{\PYZsq{}}\PY{l+s+s1}{out\PYZus{}prncp}\PY{l+s+s1}{\PYZsq{}}\PY{p}{,}\PY{l+s+s1}{\PYZsq{}}\PY{l+s+s1}{out\PYZus{}prncp\PYZus{}inv}\PY{l+s+s1}{\PYZsq{}}\PY{p}{,}\PY{l+s+s1}{\PYZsq{}}\PY{l+s+s1}{total\PYZus{}pymnt}\PY{l+s+s1}{\PYZsq{}}\PY{p}{,} \PYZbs{}
                        \PY{l+s+s1}{\PYZsq{}}\PY{l+s+s1}{total\PYZus{}pymnt\PYZus{}inv}\PY{l+s+s1}{\PYZsq{}}\PY{p}{,}\PY{l+s+s1}{\PYZsq{}}\PY{l+s+s1}{total\PYZus{}rec\PYZus{}prncp}\PY{l+s+s1}{\PYZsq{}}\PY{p}{,}\PY{l+s+s1}{\PYZsq{}}\PY{l+s+s1}{total\PYZus{}rec\PYZus{}int}\PY{l+s+s1}{\PYZsq{}}\PY{p}{,} \PYZbs{}
                        \PY{l+s+s1}{\PYZsq{}}\PY{l+s+s1}{last\PYZus{}pymnt\PYZus{}amnt}\PY{l+s+s1}{\PYZsq{}}\PY{p}{]}
        
        \PY{c+c1}{\PYZsh{} Impute NaNs with zero}
        \PY{n}{fillWithZero} \PY{o}{=} \PY{p}{[}\PY{l+s+s1}{\PYZsq{}}\PY{l+s+s1}{delinq\PYZus{}2yrs}\PY{l+s+s1}{\PYZsq{}}\PY{p}{,}\PY{l+s+s1}{\PYZsq{}}\PY{l+s+s1}{mths\PYZus{}since\PYZus{}last\PYZus{}delinq}\PY{l+s+s1}{\PYZsq{}}\PY{p}{,} \PYZbs{}
                        \PY{l+s+s1}{\PYZsq{}}\PY{l+s+s1}{mths\PYZus{}since\PYZus{}last\PYZus{}record}\PY{l+s+s1}{\PYZsq{}}\PY{p}{,}\PY{l+s+s1}{\PYZsq{}}\PY{l+s+s1}{pub\PYZus{}rec}\PY{l+s+s1}{\PYZsq{}}\PY{p}{,}\PY{l+s+s1}{\PYZsq{}}\PY{l+s+s1}{total\PYZus{}rec\PYZus{}late\PYZus{}fee}\PY{l+s+s1}{\PYZsq{}}\PY{p}{,} \PYZbs{}
                        \PY{l+s+s1}{\PYZsq{}}\PY{l+s+s1}{recoveries}\PY{l+s+s1}{\PYZsq{}}\PY{p}{,}\PY{l+s+s1}{\PYZsq{}}\PY{l+s+s1}{collection\PYZus{}recovery\PYZus{}fee}\PY{l+s+s1}{\PYZsq{}}\PY{p}{]}
        
        \PY{c+c1}{\PYZsh{} Perform imputation}
        \PY{n}{im} \PY{o}{=} \PY{n}{pp}\PY{o}{.}\PY{n}{Imputer}\PY{p}{(}\PY{n}{strategy}\PY{o}{=}\PY{l+s+s1}{\PYZsq{}}\PY{l+s+s1}{mean}\PY{l+s+s1}{\PYZsq{}}\PY{p}{)}   
        \PY{n}{data}\PY{o}{.}\PY{n}{loc}\PY{p}{[}\PY{p}{:}\PY{p}{,}\PY{n}{fillWithMean}\PY{p}{]} \PY{o}{=} \PY{n}{im}\PY{o}{.}\PY{n}{fit\PYZus{}transform}\PY{p}{(}\PY{n}{data}\PY{p}{[}\PY{n}{fillWithMean}\PY{p}{]}\PY{p}{)}
        
        \PY{n}{data}\PY{o}{.}\PY{n}{loc}\PY{p}{[}\PY{p}{:}\PY{p}{,}\PY{n}{fillWithZero}\PY{p}{]} \PY{o}{=} \PY{n}{data}\PY{o}{.}\PY{n}{loc}\PY{p}{[}\PY{p}{:}\PY{p}{,}\PY{n}{fillWithZero}\PY{p}{]}\PY{o}{.}\PY{n}{fillna}\PY{p}{(}\PY{n}{value}\PY{o}{=}\PY{l+m+mi}{0}\PY{p}{,}\PY{n}{axis}\PY{o}{=}\PY{l+m+mi}{1}\PY{p}{)}
\end{Verbatim}


    Now, let's check the NaN counts again:

    \begin{Verbatim}[commandchars=\\\{\}]
{\color{incolor}In [{\color{incolor}10}]:} \PY{c+c1}{\PYZsh{} Check for NaNs one last time}
         \PY{n}{nanCounter} \PY{o}{=} \PY{n}{np}\PY{o}{.}\PY{n}{isnan}\PY{p}{(}\PY{n}{data}\PY{o}{.}\PY{n}{loc}\PY{p}{[}\PY{p}{:}\PY{p}{,}\PY{n}{numericalFeats}\PY{p}{]}\PY{p}{)}\PY{o}{.}\PY{n}{sum}\PY{p}{(}\PY{p}{)}
         \PY{n}{nanCounter}
\end{Verbatim}


\begin{Verbatim}[commandchars=\\\{\}]
{\color{outcolor}Out[{\color{outcolor}10}]:} loan\_amnt                  0
         funded\_amnt                0
         funded\_amnt\_inv            0
         term                       0
         int\_rate                   0
         installment                0
         emp\_length                 0
         annual\_inc                 0
         dti                        0
         delinq\_2yrs                0
         mths\_since\_last\_delinq     0
         mths\_since\_last\_record     0
         open\_acc                   0
         pub\_rec                    0
         revol\_bal                  0
         revol\_util                 0
         total\_acc                  0
         out\_prncp                  0
         out\_prncp\_inv              0
         total\_pymnt                0
         total\_pymnt\_inv            0
         total\_rec\_prncp            0
         total\_rec\_int              0
         total\_rec\_late\_fee         0
         recoveries                 0
         collection\_recovery\_fee    0
         last\_pymnt\_amnt            0
         dtype: int64
\end{Verbatim}
            
    The data are now complete.

\subsubsection{Engineer Features}\label{engineer-features}

Let's add a few more features to the dataset. The new features are
ratios between the loan amount, revolving balance, payments, and the
borrower's annual income:

    \begin{Verbatim}[commandchars=\\\{\}]
{\color{incolor}In [{\color{incolor}11}]:} \PY{c+c1}{\PYZsh{} Feature engineering}
         \PY{n}{data}\PY{p}{[}\PY{l+s+s1}{\PYZsq{}}\PY{l+s+s1}{installmentOverLoanAmnt}\PY{l+s+s1}{\PYZsq{}}\PY{p}{]} \PY{o}{=} \PY{n}{data}\PY{o}{.}\PY{n}{installment}\PY{o}{/}\PY{n}{data}\PY{o}{.}\PY{n}{loan\PYZus{}amnt}
         \PY{n}{data}\PY{p}{[}\PY{l+s+s1}{\PYZsq{}}\PY{l+s+s1}{loanAmntOverIncome}\PY{l+s+s1}{\PYZsq{}}\PY{p}{]} \PY{o}{=} \PY{n}{data}\PY{o}{.}\PY{n}{loan\PYZus{}amnt}\PY{o}{/}\PY{n}{data}\PY{o}{.}\PY{n}{annual\PYZus{}inc}
         \PY{n}{data}\PY{p}{[}\PY{l+s+s1}{\PYZsq{}}\PY{l+s+s1}{revol\PYZus{}balOverIncome}\PY{l+s+s1}{\PYZsq{}}\PY{p}{]} \PY{o}{=} \PY{n}{data}\PY{o}{.}\PY{n}{revol\PYZus{}bal}\PY{o}{/}\PY{n}{data}\PY{o}{.}\PY{n}{annual\PYZus{}inc}
         \PY{n}{data}\PY{p}{[}\PY{l+s+s1}{\PYZsq{}}\PY{l+s+s1}{totalPymntOverIncome}\PY{l+s+s1}{\PYZsq{}}\PY{p}{]} \PY{o}{=} \PY{n}{data}\PY{o}{.}\PY{n}{total\PYZus{}pymnt}\PY{o}{/}\PY{n}{data}\PY{o}{.}\PY{n}{annual\PYZus{}inc}
         \PY{n}{data}\PY{p}{[}\PY{l+s+s1}{\PYZsq{}}\PY{l+s+s1}{totalPymntInvOverIncome}\PY{l+s+s1}{\PYZsq{}}\PY{p}{]} \PY{o}{=} \PY{n}{data}\PY{o}{.}\PY{n}{total\PYZus{}pymnt\PYZus{}inv}\PY{o}{/}\PY{n}{data}\PY{o}{.}\PY{n}{annual\PYZus{}inc}
         \PY{n}{data}\PY{p}{[}\PY{l+s+s1}{\PYZsq{}}\PY{l+s+s1}{totalRecPrncpOverIncome}\PY{l+s+s1}{\PYZsq{}}\PY{p}{]} \PY{o}{=} \PY{n}{data}\PY{o}{.}\PY{n}{total\PYZus{}rec\PYZus{}prncp}\PY{o}{/}\PY{n}{data}\PY{o}{.}\PY{n}{annual\PYZus{}inc}
         \PY{n}{data}\PY{p}{[}\PY{l+s+s1}{\PYZsq{}}\PY{l+s+s1}{totalRecIncOverIncome}\PY{l+s+s1}{\PYZsq{}}\PY{p}{]} \PY{o}{=} \PY{n}{data}\PY{o}{.}\PY{n}{total\PYZus{}rec\PYZus{}int}\PY{o}{/}\PY{n}{data}\PY{o}{.}\PY{n}{annual\PYZus{}inc}
         
         \PY{n}{newFeats} \PY{o}{=} \PY{p}{[}\PY{l+s+s1}{\PYZsq{}}\PY{l+s+s1}{installmentOverLoanAmnt}\PY{l+s+s1}{\PYZsq{}}\PY{p}{,}\PY{l+s+s1}{\PYZsq{}}\PY{l+s+s1}{loanAmntOverIncome}\PY{l+s+s1}{\PYZsq{}}\PY{p}{,} \PYZbs{}
                     \PY{l+s+s1}{\PYZsq{}}\PY{l+s+s1}{revol\PYZus{}balOverIncome}\PY{l+s+s1}{\PYZsq{}}\PY{p}{,}\PY{l+s+s1}{\PYZsq{}}\PY{l+s+s1}{totalPymntOverIncome}\PY{l+s+s1}{\PYZsq{}}\PY{p}{,} \PYZbs{}
                    \PY{l+s+s1}{\PYZsq{}}\PY{l+s+s1}{totalPymntInvOverIncome}\PY{l+s+s1}{\PYZsq{}}\PY{p}{,}\PY{l+s+s1}{\PYZsq{}}\PY{l+s+s1}{totalRecPrncpOverIncome}\PY{l+s+s1}{\PYZsq{}}\PY{p}{,} \PYZbs{}
                     \PY{l+s+s1}{\PYZsq{}}\PY{l+s+s1}{totalRecIncOverIncome}\PY{l+s+s1}{\PYZsq{}}\PY{p}{]}
\end{Verbatim}


    \subsubsection{Select Final Set of Features and Perform
Scaling}\label{select-final-set-of-features-and-perform-scaling}

We need to combine the selected features into a dataframe and scale the
features for the clustering:

    \begin{Verbatim}[commandchars=\\\{\}]
{\color{incolor}In [{\color{incolor}13}]:} \PY{c+c1}{\PYZsh{} Select features for training}
         \PY{n}{numericalPlusNewFeats} \PY{o}{=} \PY{n}{numericalFeats}\PY{o}{+}\PY{n}{newFeats}
         \PY{n}{X\PYZus{}train} \PY{o}{=} \PY{n}{data}\PY{o}{.}\PY{n}{loc}\PY{p}{[}\PY{p}{:}\PY{p}{,}\PY{n}{numericalPlusNewFeats}\PY{p}{]}
         
         \PY{c+c1}{\PYZsh{} Scale data}
         \PY{n}{sX} \PY{o}{=} \PY{n}{pp}\PY{o}{.}\PY{n}{StandardScaler}\PY{p}{(}\PY{p}{)}
         \PY{n}{X\PYZus{}train}\PY{o}{.}\PY{n}{loc}\PY{p}{[}\PY{p}{:}\PY{p}{,}\PY{p}{:}\PY{p}{]} \PY{o}{=} \PY{n}{sX}\PY{o}{.}\PY{n}{fit\PYZus{}transform}\PY{p}{(}\PY{n}{X\PYZus{}train}\PY{p}{)}
         \PY{n}{X\PYZus{}train}\PY{o}{.}\PY{n}{columns}
\end{Verbatim}


\begin{Verbatim}[commandchars=\\\{\}]
{\color{outcolor}Out[{\color{outcolor}13}]:} Index(['loan\_amnt', 'funded\_amnt', 'funded\_amnt\_inv', 'term', 'int\_rate',
                'installment', 'emp\_length', 'annual\_inc', 'dti', 'delinq\_2yrs',
                'mths\_since\_last\_delinq', 'mths\_since\_last\_record', 'open\_acc',
                'pub\_rec', 'revol\_bal', 'revol\_util', 'total\_acc', 'out\_prncp',
                'out\_prncp\_inv', 'total\_pymnt', 'total\_pymnt\_inv', 'total\_rec\_prncp',
                'total\_rec\_int', 'total\_rec\_late\_fee', 'recoveries',
                'collection\_recovery\_fee', 'last\_pymnt\_amnt', 'installmentOverLoanAmnt',
                'loanAmntOverIncome', 'revol\_balOverIncome', 'totalPymntOverIncome',
                'totalPymntInvOverIncome', 'totalRecPrncpOverIncome',
                'totalRecIncOverIncome'],
               dtype='object')
\end{Verbatim}
            
    \subsubsection{Designate the Labels for the
Evaluation}\label{designate-the-labels-for-the-evaluation}

Clustering is an unsupervised approach and does not use labels. However,
to evaluate the performance of the clustering algorithms, we will use
loan grades as a sort of a label.

The loan grades are graded by letters, with "A" being the highest grade
and "G" the lowest (riskiest).

    \begin{Verbatim}[commandchars=\\\{\}]
{\color{incolor}In [{\color{incolor}14}]:} \PY{c+c1}{\PYZsh{} Designate labels for evaluation}
         \PY{n}{labels} \PY{o}{=} \PY{n}{data}\PY{o}{.}\PY{n}{grade}
         \PY{n}{labels}\PY{o}{.}\PY{n}{unique}\PY{p}{(}\PY{p}{)}
\end{Verbatim}


\begin{Verbatim}[commandchars=\\\{\}]
{\color{outcolor}Out[{\color{outcolor}14}]:} array(['B', 'C', 'A', 'E', 'F', 'D', 'G', nan], dtype=object)
\end{Verbatim}
            
    There are a few NaNs in the grade feature. We will fill those with a
distinct label "Z" and use the \texttt{LabelEncoder} to transform the
letter grades to a numerical format.

    \begin{Verbatim}[commandchars=\\\{\}]
{\color{incolor}In [{\color{incolor}15}]:} \PY{c+c1}{\PYZsh{} Fill missing labels}
         \PY{n}{labels} \PY{o}{=} \PY{n}{labels}\PY{o}{.}\PY{n}{fillna}\PY{p}{(}\PY{n}{value}\PY{o}{=}\PY{l+s+s2}{\PYZdq{}}\PY{l+s+s2}{Z}\PY{l+s+s2}{\PYZdq{}}\PY{p}{)}
         
         \PY{c+c1}{\PYZsh{} Convert labels to numerical values}
         \PY{n}{lbl} \PY{o}{=} \PY{n}{pp}\PY{o}{.}\PY{n}{LabelEncoder}\PY{p}{(}\PY{p}{)}
         \PY{n}{lbl}\PY{o}{.}\PY{n}{fit}\PY{p}{(}\PY{n+nb}{list}\PY{p}{(}\PY{n}{labels}\PY{o}{.}\PY{n}{values}\PY{p}{)}\PY{p}{)}
         \PY{n}{labels} \PY{o}{=} \PY{n}{pd}\PY{o}{.}\PY{n}{Series}\PY{p}{(}\PY{n}{data}\PY{o}{=}\PY{n}{lbl}\PY{o}{.}\PY{n}{transform}\PY{p}{(}\PY{n}{labels}\PY{o}{.}\PY{n}{values}\PY{p}{)}\PY{p}{,} \PY{n}{name}\PY{o}{=}\PY{l+s+s2}{\PYZdq{}}\PY{l+s+s2}{grade}\PY{l+s+s2}{\PYZdq{}}\PY{p}{)}
         
         \PY{c+c1}{\PYZsh{} Store as y\PYZus{}train}
         \PY{n}{y\PYZus{}train} \PY{o}{=} \PY{n}{labels}
\end{Verbatim}


    \begin{Verbatim}[commandchars=\\\{\}]
{\color{incolor}In [{\color{incolor}17}]:} \PY{n}{labelsOriginalVSNew} \PY{o}{=} \PY{n}{pd}\PY{o}{.}\PY{n}{concat}\PY{p}{(}\PY{p}{[}\PY{n}{labels}\PY{p}{,} \PY{n}{data}\PY{o}{.}\PY{n}{grade}\PY{p}{]}\PY{p}{,}\PY{n}{axis}\PY{o}{=}\PY{l+m+mi}{1}\PY{p}{)}
         \PY{n}{labelsOriginalVSNew}\PY{o}{.}\PY{n}{head}\PY{p}{(}\PY{p}{)}
\end{Verbatim}


\begin{Verbatim}[commandchars=\\\{\}]
{\color{outcolor}Out[{\color{outcolor}17}]:}   grade grade
         0     1     B
         1     2     C
         2     2     C
         3     2     C
         4     1     B
\end{Verbatim}
            
    Let's see what interest rates are applied to what lean grades:

    \begin{Verbatim}[commandchars=\\\{\}]
{\color{incolor}In [{\color{incolor}26}]:} \PY{c+c1}{\PYZsh{} Compare loan grades with interest rates}
         \PY{n}{interestAndGrade} \PY{o}{=} \PY{n}{pd}\PY{o}{.}\PY{n}{DataFrame}\PY{p}{(}\PY{n}{data}\PY{o}{=}\PY{p}{[}\PY{n}{data}\PY{o}{.}\PY{n}{int\PYZus{}rate}\PY{p}{,}\PY{n}{labels}\PY{p}{]}\PY{p}{)}
         \PY{n}{interestAndGrade} \PY{o}{=} \PY{n}{interestAndGrade}\PY{o}{.}\PY{n}{T}
         \PY{n}{interestAndGrade}\PY{o}{.}\PY{n}{groupby}\PY{p}{(}\PY{l+s+s2}{\PYZdq{}}\PY{l+s+s2}{grade}\PY{l+s+s2}{\PYZdq{}}\PY{p}{)}\PY{o}{.}\PY{n}{mean}\PY{p}{(}\PY{p}{)}
\end{Verbatim}


\begin{Verbatim}[commandchars=\\\{\}]
{\color{outcolor}Out[{\color{outcolor}26}]:}           int\_rate
         grade             
         0.0     734.270844
         1.0    1101.420857
         2.0    1349.988902
         3.0    1557.714927
         4.0    1737.676783
         5.0    1926.530361
         6.0    2045.125000
         7.0    1216.501563
\end{Verbatim}
            
    \subsubsection{Goodness of the Clusters}\label{goodness-of-the-clusters}

We have prepared a training dataset with 34 features and a label set
with numerical loan grades. The labels will not be used for the
training, they will only be employed to evaluate the results of the
training.

For this evaluation, we will need a special function. The performance of
the clustering application can be called good when the separation groups
together borrowers that are similar to each other and dissimilar to
members of other groups. We assume that similar borrowers will not only
have similar lending profiles but also similar loan grades. This means,
borrowers within a cluster must have the same loan grade. To evaluate
the goodness of the cluster we will use the \texttt{y\_train} label
dataset to see what proportion of the most popular loan grade the
cluster has (how homogenous it is).

    \begin{Verbatim}[commandchars=\\\{\}]
{\color{incolor}In [{\color{incolor}27}]:} \PY{k}{def} \PY{n+nf}{analyzeCluster}\PY{p}{(}\PY{n}{clusterDF}\PY{p}{,} \PY{n}{labelsDF}\PY{p}{)}\PY{p}{:}
             \PY{c+c1}{\PYZsh{} Calculate number of cluster members}
             \PY{n}{countByCluster} \PY{o}{=} \PYZbs{}
                 \PY{n}{pd}\PY{o}{.}\PY{n}{DataFrame}\PY{p}{(}\PY{n}{data}\PY{o}{=}\PY{n}{clusterDF}\PY{p}{[}\PY{l+s+s1}{\PYZsq{}}\PY{l+s+s1}{cluster}\PY{l+s+s1}{\PYZsq{}}\PY{p}{]}\PY{o}{.}\PY{n}{value\PYZus{}counts}\PY{p}{(}\PY{p}{)}\PY{p}{)}
             \PY{n}{countByCluster}\PY{o}{.}\PY{n}{reset\PYZus{}index}\PY{p}{(}\PY{n}{inplace}\PY{o}{=}\PY{k+kc}{True}\PY{p}{,}\PY{n}{drop}\PY{o}{=}\PY{k+kc}{False}\PY{p}{)}
             \PY{n}{countByCluster}\PY{o}{.}\PY{n}{columns} \PY{o}{=} \PY{p}{[}\PY{l+s+s1}{\PYZsq{}}\PY{l+s+s1}{cluster}\PY{l+s+s1}{\PYZsq{}}\PY{p}{,}\PY{l+s+s1}{\PYZsq{}}\PY{l+s+s1}{clusterCount}\PY{l+s+s1}{\PYZsq{}}\PY{p}{]}
         
             \PY{c+c1}{\PYZsh{} for each point, store its true label and its cluster assignment}
             \PY{n}{preds} \PY{o}{=} \PY{n}{pd}\PY{o}{.}\PY{n}{concat}\PY{p}{(}\PY{p}{[}\PY{n}{labelsDF}\PY{p}{,}\PY{n}{clusterDF}\PY{p}{]}\PY{p}{,} \PY{n}{axis}\PY{o}{=}\PY{l+m+mi}{1}\PY{p}{)}
             \PY{n}{preds}\PY{o}{.}\PY{n}{columns} \PY{o}{=} \PY{p}{[}\PY{l+s+s1}{\PYZsq{}}\PY{l+s+s1}{trueLabel}\PY{l+s+s1}{\PYZsq{}}\PY{p}{,}\PY{l+s+s1}{\PYZsq{}}\PY{l+s+s1}{cluster}\PY{l+s+s1}{\PYZsq{}}\PY{p}{]}
             
             \PY{c+c1}{\PYZsh{} number of each label occurrences}
             \PY{n}{countByLabel} \PY{o}{=} \PY{n}{pd}\PY{o}{.}\PY{n}{DataFrame}\PY{p}{(}\PY{n}{data}\PY{o}{=}\PY{n}{preds}\PY{o}{.}\PY{n}{groupby}\PY{p}{(}\PY{l+s+s1}{\PYZsq{}}\PY{l+s+s1}{trueLabel}\PY{l+s+s1}{\PYZsq{}}\PY{p}{)}\PY{o}{.}\PY{n}{count}\PY{p}{(}\PY{p}{)}\PY{p}{)}
         
             \PY{c+c1}{\PYZsh{} group by clusters and let the count }
             \PY{c+c1}{\PYZsh{} of the most popular label in the cluster remain}
             \PY{n}{countMostFreq} \PY{o}{=} \PY{n}{pd}\PY{o}{.}\PY{n}{DataFrame}\PY{p}{(}\PY{n}{data}\PY{o}{=}\PY{n}{preds}\PY{o}{.}\PY{n}{groupby}\PY{p}{(}\PY{l+s+s1}{\PYZsq{}}\PY{l+s+s1}{cluster}\PY{l+s+s1}{\PYZsq{}}\PY{p}{)}\PY{o}{.}\PY{n}{agg}\PY{p}{(} \PYZbs{}
                 \PY{k}{lambda} \PY{n}{x}\PY{p}{:}\PY{n}{x}\PY{o}{.}\PY{n}{value\PYZus{}counts}\PY{p}{(}\PY{p}{)}\PY{o}{.}\PY{n}{iloc}\PY{p}{[}\PY{l+m+mi}{0}\PY{p}{]}\PY{p}{)}\PY{p}{)}
             \PY{n}{countMostFreq}\PY{o}{.}\PY{n}{reset\PYZus{}index}\PY{p}{(}\PY{n}{inplace}\PY{o}{=}\PY{k+kc}{True}\PY{p}{,}\PY{n}{drop}\PY{o}{=}\PY{k+kc}{False}\PY{p}{)}
             \PY{n}{countMostFreq}\PY{o}{.}\PY{n}{columns} \PY{o}{=} \PY{p}{[}\PY{l+s+s1}{\PYZsq{}}\PY{l+s+s1}{cluster}\PY{l+s+s1}{\PYZsq{}}\PY{p}{,}\PY{l+s+s1}{\PYZsq{}}\PY{l+s+s1}{countMostFrequent}\PY{l+s+s1}{\PYZsq{}}\PY{p}{]}
             
             \PY{c+c1}{\PYZsh{} merge the count of the most popular label in }
             \PY{c+c1}{\PYZsh{} the cluster with the cluster size}
             \PY{n}{accuracyDF} \PY{o}{=} \PY{n}{countMostFreq}\PY{o}{.}\PY{n}{merge}\PY{p}{(}\PY{n}{countByCluster}\PY{p}{,} \PYZbs{}
                 \PY{n}{left\PYZus{}on}\PY{o}{=}\PY{l+s+s2}{\PYZdq{}}\PY{l+s+s2}{cluster}\PY{l+s+s2}{\PYZdq{}}\PY{p}{,}\PY{n}{right\PYZus{}on}\PY{o}{=}\PY{l+s+s2}{\PYZdq{}}\PY{l+s+s2}{cluster}\PY{l+s+s2}{\PYZdq{}}\PY{p}{)}
             
             \PY{n}{overallAccuracy} \PY{o}{=} \PY{n}{accuracyDF}\PY{o}{.}\PY{n}{countMostFrequent}\PY{o}{.}\PY{n}{sum}\PY{p}{(}\PY{p}{)}\PY{o}{/} \PYZbs{}
                 \PY{n}{accuracyDF}\PY{o}{.}\PY{n}{clusterCount}\PY{o}{.}\PY{n}{sum}\PY{p}{(}\PY{p}{)}
             
             \PY{n}{accuracyByLabel} \PY{o}{=} \PY{n}{accuracyDF}\PY{o}{.}\PY{n}{countMostFrequent}\PY{o}{/} \PYZbs{}
                 \PY{n}{accuracyDF}\PY{o}{.}\PY{n}{clusterCount}
             
             \PY{k}{return} \PY{n}{countByCluster}\PY{p}{,} \PY{n}{countByLabel}\PY{p}{,} \PY{n}{countMostFreq}\PY{p}{,} \PYZbs{}
                 \PY{n}{accuracyDF}\PY{p}{,} \PY{n}{overallAccuracy}\PY{p}{,} \PY{n}{accuracyByLabel}
\end{Verbatim}


    \subsection{k-Means Application}\label{k-means-application}

First, we will apply the \emph{k-means} algorithm that minimizes the
cluster inertia. We have to set the number of clusters \emph{k}. To find
the optimum value for \emph{k} we will experiment with a range of 10 to
30 clusters.

    \begin{Verbatim}[commandchars=\\\{\}]
{\color{incolor}In [{\color{incolor}28}]:} \PY{k+kn}{from} \PY{n+nn}{sklearn}\PY{n+nn}{.}\PY{n+nn}{cluster} \PY{k}{import} \PY{n}{KMeans}
         
         \PY{n}{n\PYZus{}clusters} \PY{o}{=} \PY{l+m+mi}{10}
         \PY{n}{n\PYZus{}init} \PY{o}{=} \PY{l+m+mi}{10}
         \PY{n}{max\PYZus{}iter} \PY{o}{=} \PY{l+m+mi}{300}
         \PY{n}{tol} \PY{o}{=} \PY{l+m+mf}{0.0001}
         \PY{n}{random\PYZus{}state} \PY{o}{=} \PY{l+m+mi}{2018}
         \PY{n}{n\PYZus{}jobs} \PY{o}{=} \PY{l+m+mi}{2}
         
         \PY{n}{kmeans} \PY{o}{=} \PY{n}{KMeans}\PY{p}{(}\PY{n}{n\PYZus{}clusters}\PY{o}{=}\PY{n}{n\PYZus{}clusters}\PY{p}{,} \PY{n}{n\PYZus{}init}\PY{o}{=}\PY{n}{n\PYZus{}init}\PY{p}{,} \PYZbs{}
                         \PY{n}{max\PYZus{}iter}\PY{o}{=}\PY{n}{max\PYZus{}iter}\PY{p}{,} \PY{n}{tol}\PY{o}{=}\PY{n}{tol}\PY{p}{,} \PYZbs{}
                         \PY{n}{random\PYZus{}state}\PY{o}{=}\PY{n}{random\PYZus{}state}\PY{p}{,} \PY{n}{n\PYZus{}jobs}\PY{o}{=}\PY{n}{n\PYZus{}jobs}\PY{p}{)}
         
         \PY{n}{kMeans\PYZus{}inertia} \PY{o}{=} \PY{n}{pd}\PY{o}{.}\PY{n}{DataFrame}\PY{p}{(}\PY{n}{data}\PY{o}{=}\PY{p}{[}\PY{p}{]}\PY{p}{,}\PY{n}{index}\PY{o}{=}\PY{n+nb}{range}\PY{p}{(}\PY{l+m+mi}{10}\PY{p}{,}\PY{l+m+mi}{31}\PY{p}{)}\PY{p}{,} \PYZbs{}
                                       \PY{n}{columns}\PY{o}{=}\PY{p}{[}\PY{l+s+s1}{\PYZsq{}}\PY{l+s+s1}{inertia}\PY{l+s+s1}{\PYZsq{}}\PY{p}{]}\PY{p}{)}
         
         \PY{n}{overallAccuracy\PYZus{}kMeansDF} \PY{o}{=} \PY{n}{pd}\PY{o}{.}\PY{n}{DataFrame}\PY{p}{(}\PY{n}{data}\PY{o}{=}\PY{p}{[}\PY{p}{]}\PY{p}{,} \PYZbs{}
             \PY{n}{index}\PY{o}{=}\PY{n+nb}{range}\PY{p}{(}\PY{l+m+mi}{10}\PY{p}{,}\PY{l+m+mi}{31}\PY{p}{)}\PY{p}{,}\PY{n}{columns}\PY{o}{=}\PY{p}{[}\PY{l+s+s1}{\PYZsq{}}\PY{l+s+s1}{overallAccuracy}\PY{l+s+s1}{\PYZsq{}}\PY{p}{]}\PY{p}{)}
         
         \PY{k}{for} \PY{n}{n\PYZus{}clusters} \PY{o+ow}{in} \PY{n+nb}{range}\PY{p}{(}\PY{l+m+mi}{10}\PY{p}{,}\PY{l+m+mi}{31}\PY{p}{)}\PY{p}{:}
             \PY{n}{kmeans} \PY{o}{=} \PY{n}{KMeans}\PY{p}{(}\PY{n}{n\PYZus{}clusters}\PY{o}{=}\PY{n}{n\PYZus{}clusters}\PY{p}{,} \PY{n}{n\PYZus{}init}\PY{o}{=}\PY{n}{n\PYZus{}init}\PY{p}{,} \PYZbs{}
                             \PY{n}{max\PYZus{}iter}\PY{o}{=}\PY{n}{max\PYZus{}iter}\PY{p}{,} \PY{n}{tol}\PY{o}{=}\PY{n}{tol}\PY{p}{,} \PYZbs{}
                             \PY{n}{random\PYZus{}state}\PY{o}{=}\PY{n}{random\PYZus{}state}\PY{p}{,} \PY{n}{n\PYZus{}jobs}\PY{o}{=}\PY{n}{n\PYZus{}jobs}\PY{p}{)}
         
             \PY{n}{kmeans}\PY{o}{.}\PY{n}{fit}\PY{p}{(}\PY{n}{X\PYZus{}train}\PY{p}{)}
             \PY{n}{kMeans\PYZus{}inertia}\PY{o}{.}\PY{n}{loc}\PY{p}{[}\PY{n}{n\PYZus{}clusters}\PY{p}{]} \PY{o}{=} \PY{n}{kmeans}\PY{o}{.}\PY{n}{inertia\PYZus{}}
             \PY{n}{X\PYZus{}train\PYZus{}kmeansClustered} \PY{o}{=} \PY{n}{kmeans}\PY{o}{.}\PY{n}{predict}\PY{p}{(}\PY{n}{X\PYZus{}train}\PY{p}{)}
             \PY{n}{X\PYZus{}train\PYZus{}kmeansClustered} \PY{o}{=} \PY{n}{pd}\PY{o}{.}\PY{n}{DataFrame}\PY{p}{(}\PY{n}{data}\PY{o}{=} \PYZbs{}
                 \PY{n}{X\PYZus{}train\PYZus{}kmeansClustered}\PY{p}{,} \PY{n}{index}\PY{o}{=}\PY{n}{X\PYZus{}train}\PY{o}{.}\PY{n}{index}\PY{p}{,} \PYZbs{}
                 \PY{n}{columns}\PY{o}{=}\PY{p}{[}\PY{l+s+s1}{\PYZsq{}}\PY{l+s+s1}{cluster}\PY{l+s+s1}{\PYZsq{}}\PY{p}{]}\PY{p}{)}
             
             \PY{n}{countByCluster\PYZus{}kMeans}\PY{p}{,} \PY{n}{countByLabel\PYZus{}kMeans}\PY{p}{,} \PYZbs{}
             \PY{n}{countMostFreq\PYZus{}kMeans}\PY{p}{,} \PY{n}{accuracyDF\PYZus{}kMeans}\PY{p}{,} \PYZbs{}
             \PY{n}{overallAccuracy\PYZus{}kMeans}\PY{p}{,} \PY{n}{accuracyByLabel\PYZus{}kMeans} \PY{o}{=} \PYZbs{}
             \PY{n}{analyzeCluster}\PY{p}{(}\PY{n}{X\PYZus{}train\PYZus{}kmeansClustered}\PY{p}{,} \PY{n}{y\PYZus{}train}\PY{p}{)}
             
             \PY{n}{overallAccuracy\PYZus{}kMeansDF}\PY{o}{.}\PY{n}{loc}\PY{p}{[}\PY{n}{n\PYZus{}clusters}\PY{p}{]} \PY{o}{=} \PYZbs{}
                 \PY{n}{overallAccuracy\PYZus{}kMeans}
\end{Verbatim}


    \begin{Verbatim}[commandchars=\\\{\}]
{\color{incolor}In [{\color{incolor}30}]:} \PY{n}{overallAccuracy\PYZus{}kMeansDF}\PY{o}{.}\PY{n}{plot}\PY{p}{(}\PY{p}{)}
\end{Verbatim}


\begin{Verbatim}[commandchars=\\\{\}]
{\color{outcolor}Out[{\color{outcolor}30}]:} <matplotlib.axes.\_subplots.AxesSubplot at 0x294c43f4048>
\end{Verbatim}
            
    \begin{center}
    \adjustimage{max size={0.9\linewidth}{0.9\paperheight}}{output_30_1.png}
    \end{center}
    { \hspace*{\fill} \\}
    
    One can see that accuracy is best around 30 clusters where it reaches
39\%. This means that in clusters on average we will find 39\% of
borrowers with the same profile and the rest of the cluster members will
have other loan grades.

This is how the accuracy per cluster looks like at 30 clusters:

    \begin{Verbatim}[commandchars=\\\{\}]
{\color{incolor}In [{\color{incolor}74}]:} \PY{n}{accuracyByLabel\PYZus{}kMeans}
\end{Verbatim}


\begin{Verbatim}[commandchars=\\\{\}]
{\color{outcolor}Out[{\color{outcolor}74}]:} 0     0.326633
         1     0.258993
         2     0.292240
         3     0.234242
         4     0.388794
         5     0.325654
         6     0.303797
         7     0.762116
         8     0.222222
         9     0.391381
         10    0.292910
         11    0.317533
         12    0.206897
         13    0.312709
         14    0.345233
         15    0.682208
         16    0.327250
         17    0.366605
         18    0.234783
         19    0.288757
         20    0.500000
         21    0.375466
         22    0.332203
         23    0.252252
         24    0.338509
         25    0.232000
         26    0.464418
         27    0.261583
         28    0.376327
         29    0.269129
         dtype: float64
\end{Verbatim}
            
    The accuracy varies a lot between clusters with the most homogenous
cluster (7) having 76\% and the least homogenous cluster (12) only 21\%.
This is not a good enough result.

We could try collecting more data or engineer more features to improve
the result. It is also possible that \emph{k-means} is just not a good
algorithm for this dataset and we should try something else.

    \subsection{Hierarchical Clustering
Algorithms}\label{hierarchical-clustering-algorithms}

After applying the hierarchical clustering to the dataset we will have
to decide where to cut the tree.

    \begin{Verbatim}[commandchars=\\\{\}]
{\color{incolor}In [{\color{incolor}75}]:} \PY{k+kn}{import} \PY{n+nn}{fastcluster}
         \PY{k+kn}{from} \PY{n+nn}{scipy}\PY{n+nn}{.}\PY{n+nn}{cluster}\PY{n+nn}{.}\PY{n+nn}{hierarchy} \PY{k}{import} \PY{n}{dendrogram}
         \PY{k+kn}{from} \PY{n+nn}{scipy}\PY{n+nn}{.}\PY{n+nn}{cluster}\PY{n+nn}{.}\PY{n+nn}{hierarchy} \PY{k}{import} \PY{n}{cophenet}
         \PY{k+kn}{from} \PY{n+nn}{scipy}\PY{n+nn}{.}\PY{n+nn}{spatial}\PY{n+nn}{.}\PY{n+nn}{distance} \PY{k}{import} \PY{n}{pdist}
         
         \PY{n}{Z} \PY{o}{=} \PY{n}{fastcluster}\PY{o}{.}\PY{n}{linkage\PYZus{}vector}\PY{p}{(}\PY{n}{X\PYZus{}train}\PY{p}{,} \PY{n}{method}\PY{o}{=}\PY{l+s+s1}{\PYZsq{}}\PY{l+s+s1}{ward}\PY{l+s+s1}{\PYZsq{}}\PY{p}{,} \PYZbs{}
                                        \PY{n}{metric}\PY{o}{=}\PY{l+s+s1}{\PYZsq{}}\PY{l+s+s1}{euclidean}\PY{l+s+s1}{\PYZsq{}}\PY{p}{)}
         
         \PY{n}{Z\PYZus{}dataFrame} \PY{o}{=} \PY{n}{pd}\PY{o}{.}\PY{n}{DataFrame}\PY{p}{(}\PY{n}{data}\PY{o}{=}\PY{n}{Z}\PY{p}{,}\PY{n}{columns}\PY{o}{=}\PY{p}{[}\PY{l+s+s1}{\PYZsq{}}\PY{l+s+s1}{clusterOne}\PY{l+s+s1}{\PYZsq{}}\PY{p}{,} \PYZbs{}
                         \PY{l+s+s1}{\PYZsq{}}\PY{l+s+s1}{clusterTwo}\PY{l+s+s1}{\PYZsq{}}\PY{p}{,}\PY{l+s+s1}{\PYZsq{}}\PY{l+s+s1}{distance}\PY{l+s+s1}{\PYZsq{}}\PY{p}{,}\PY{l+s+s1}{\PYZsq{}}\PY{l+s+s1}{newClusterSize}\PY{l+s+s1}{\PYZsq{}}\PY{p}{]}\PY{p}{)}
\end{Verbatim}


    The first 10 clusters look like this:

    \begin{Verbatim}[commandchars=\\\{\}]
{\color{incolor}In [{\color{incolor}78}]:} \PY{n+nb}{print}\PY{p}{(}\PY{n}{Z\PYZus{}dataFrame}\PY{o}{.}\PY{n}{iloc}\PY{p}{[}\PY{l+m+mi}{0}\PY{p}{:}\PY{l+m+mi}{10}\PY{p}{]}\PY{p}{)}
\end{Verbatim}


    \begin{Verbatim}[commandchars=\\\{\}]
   clusterOne  clusterTwo      distance  newClusterSize
0     39786.0     39787.0  0.000000e+00             2.0
1     39788.0     42542.0  0.000000e+00             3.0
2     42538.0     42539.0  0.000000e+00             2.0
3     42540.0     42544.0  0.000000e+00             3.0
4     42541.0     42545.0  3.399350e-17             4.0
5     42543.0     42546.0  5.139334e-17             7.0
6     33251.0     33261.0  1.561313e-01             2.0
7     42512.0     42535.0  3.342654e-01             2.0
8     42219.0     42316.0  3.368231e-01             2.0
9      6112.0     21928.0  3.384368e-01             2.0

    \end{Verbatim}

    And the last 10 clusters like this:

    \begin{Verbatim}[commandchars=\\\{\}]
{\color{incolor}In [{\color{incolor}81}]:} \PY{n+nb}{print}\PY{p}{(}\PY{n}{Z\PYZus{}dataFrame}\PY{o}{.}\PY{n}{iloc}\PY{p}{[}\PY{o}{\PYZhy{}}\PY{l+m+mi}{10}\PY{p}{:}\PY{p}{]}\PY{p}{)}
\end{Verbatim}


    \begin{Verbatim}[commandchars=\\\{\}]
       clusterOne  clusterTwo    distance  newClusterSize
42531     85067.0     85071.0  187.832588         15370.0
42532     85056.0     85073.0  203.212147         17995.0
42533     85057.0     85063.0  205.285993          9221.0
42534     85068.0     85072.0  207.902660          5321.0
42535     85069.0     85075.0  236.754581          9889.0
42536     85070.0     85077.0  298.587755         16786.0
42537     85058.0     85078.0  309.946867         16875.0
42538     85074.0     85079.0  375.698458         34870.0
42539     85065.0     85080.0  400.711547         37221.0
42540     85076.0     85081.0  644.047472         42542.0

    \end{Verbatim}

    Now, we need to cut the tree to get a manageable number of clusters. We
will have to find a value for \texttt{distance\_threshold} that yields
around 30 clusters. This distance is 100 and the resulting number of
clusters is 32:

    \begin{Verbatim}[commandchars=\\\{\}]
{\color{incolor}In [{\color{incolor}107}]:} \PY{k+kn}{from} \PY{n+nn}{scipy}\PY{n+nn}{.}\PY{n+nn}{cluster}\PY{n+nn}{.}\PY{n+nn}{hierarchy} \PY{k}{import} \PY{n}{fcluster}
          
          \PY{n}{distance\PYZus{}threshold} \PY{o}{=} \PY{l+m+mi}{100}
          \PY{n}{clusters} \PY{o}{=} \PY{n}{fcluster}\PY{p}{(}\PY{n}{Z}\PY{p}{,} \PY{n}{distance\PYZus{}threshold}\PY{p}{,} \PY{n}{criterion}\PY{o}{=}\PY{l+s+s1}{\PYZsq{}}\PY{l+s+s1}{distance}\PY{l+s+s1}{\PYZsq{}}\PY{p}{)}
          \PY{n}{X\PYZus{}train\PYZus{}hierClustered} \PY{o}{=} \PY{n}{pd}\PY{o}{.}\PY{n}{DataFrame}\PY{p}{(}\PY{n}{data}\PY{o}{=}\PY{n}{clusters}\PY{p}{,} \PYZbs{}
              \PY{n}{index}\PY{o}{=}\PY{n}{X\PYZus{}train}\PY{o}{.}\PY{n}{index}\PY{p}{,}\PY{n}{columns}\PY{o}{=}\PY{p}{[}\PY{l+s+s1}{\PYZsq{}}\PY{l+s+s1}{cluster}\PY{l+s+s1}{\PYZsq{}}\PY{p}{]}\PY{p}{)}
\end{Verbatim}


    \begin{Verbatim}[commandchars=\\\{\}]
{\color{incolor}In [{\color{incolor}108}]:} \PY{n+nb}{print}\PY{p}{(}\PY{l+s+s2}{\PYZdq{}}\PY{l+s+s2}{Number of distinct clusters: }\PY{l+s+s2}{\PYZdq{}}\PY{p}{,} \PYZbs{}
                \PY{n+nb}{len}\PY{p}{(}\PY{n}{X\PYZus{}train\PYZus{}hierClustered}\PY{p}{[}\PY{l+s+s1}{\PYZsq{}}\PY{l+s+s1}{cluster}\PY{l+s+s1}{\PYZsq{}}\PY{p}{]}\PY{o}{.}\PY{n}{unique}\PY{p}{(}\PY{p}{)}\PY{p}{)}\PY{p}{)}
\end{Verbatim}


    \begin{Verbatim}[commandchars=\\\{\}]
Number of distinct clusters:  32

    \end{Verbatim}

    \begin{Verbatim}[commandchars=\\\{\}]
{\color{incolor}In [{\color{incolor}109}]:} \PY{n}{countByCluster\PYZus{}hierClust}\PY{p}{,} \PY{n}{countByLabel\PYZus{}hierClust}\PY{p}{,} \PYZbs{}
              \PY{n}{countMostFreq\PYZus{}hierClust}\PY{p}{,} \PY{n}{accuracyDF\PYZus{}hierClust}\PY{p}{,} \PYZbs{}
              \PY{n}{overallAccuracy\PYZus{}hierClust}\PY{p}{,} \PY{n}{accuracyByLabel\PYZus{}hierClust} \PY{o}{=} \PYZbs{}
              \PY{n}{analyzeCluster}\PY{p}{(}\PY{n}{X\PYZus{}train\PYZus{}hierClustered}\PY{p}{,} \PY{n}{y\PYZus{}train}\PY{p}{)}
          
          \PY{n+nb}{print}\PY{p}{(}\PY{l+s+s2}{\PYZdq{}}\PY{l+s+s2}{Overall accuracy from hierarchical clustering: }\PY{l+s+s2}{\PYZdq{}}\PY{p}{,} \PYZbs{}
                \PY{n}{overallAccuracy\PYZus{}hierClust}\PY{p}{)}
\end{Verbatim}


    \begin{Verbatim}[commandchars=\\\{\}]
Overall accuracy from hierarchical clustering:  0.3651685393258427

    \end{Verbatim}

    The overall accuracy is 36\% which is a bit worse than with
\emph{k-means} clustering. Since hierarchical clustering works
differently than \emph{k-means} it can be said that the cluster
assignments can also be different.

In other words, these two algorithms may complement each other if we use
them in an ensemble.

Again, we can see that some clusters are more homogenous than others,
the top cluster is 16 with 74\% and the bottom cluster is 1 with 22\%:

    \begin{Verbatim}[commandchars=\\\{\}]
{\color{incolor}In [{\color{incolor}110}]:} \PY{n}{accuracyByLabel\PYZus{}hierClust}
\end{Verbatim}


\begin{Verbatim}[commandchars=\\\{\}]
{\color{outcolor}Out[{\color{outcolor}110}]:} 0     0.304124
          1     0.219001
          2     0.228311
          3     0.379722
          4     0.240064
          5     0.272011
          6     0.314560
          7     0.263930
          8     0.246138
          9     0.318942
          10    0.302752
          11    0.269772
          12    0.335717
          13    0.330403
          14    0.346320
          15    0.440141
          16    0.744155
          17    0.502227
          18    0.294118
          19    0.236111
          20    0.254727
          21    0.241042
          22    0.317979
          23    0.308771
          24    0.284314
          25    0.243243
          26    0.500000
          27    0.289157
          28    0.365283
          29    0.479693
          30    0.393559
          31    0.340875
          dtype: float64
\end{Verbatim}
            

    % Add a bibliography block to the postdoc
    
    
    
    \end{document}
